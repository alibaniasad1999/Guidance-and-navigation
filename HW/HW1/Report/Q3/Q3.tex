\section{سوال سوم}
در این بخش معادلات حاکم بر سیستم موقعیت‌یاب جهانی حل شده است. 
\subsection{بخش اول}
برای محاسبه مکان گیرنده با استفاده از چهار ماهواره، چهار معادله و چهار مجهول زیر حل شده است و نتایج آن در جدول  
\ref{tab:4gps_solve}
آورده شده است. سرعت نور
$
3\times10^{5}_{km/s}
$ 
در نظر گرفته شده است.
\begin{align}
	\begin{split}
		(x-A_1)^2+(y-B_1)^2+(z-C_1)^2-(c(t_1-d))^2&=0 \\
		(x-A_2)^2+(y-B_2)^2+(z-C_2)^2-(c(t_2-d))^2&=0 \\
		(x-A_3)^2+(y-B_3)^2+(z-C_3)^2-(c(t_3-d))^2&=0 \\
		(x-A_4)^2+(y-B_4)^2+(z-C_4)^2-(c(t_4-d))^2&=0 
	\end{split}
\end{align}


\begin{table}[H]
	\caption{مکان و تاخیر بدست آمده از داده‌های چهار \lr{GPS}}
	\vspace{0.2cm}
	\centering
		\begin{tabular}{ccccc}
				\hline
				  \multirow{2}{*}{\lr{delay ($\sec$)}} & \multicolumn{3}{c}{\lr{position (Km)}} & \multirow{2}{*}{\lr{solution number}}\\ 
				\cmidrule(lr){2-4}		
				& \lr{z} & \lr{y} & \lr{x} & \\  \hline		
				\lr{70.81} & \lr{-164331.37} & \lr{-51463.07} & \lr{2810343.11} & \lr{1}\\
				\lr{-56.06} & \lr{-600793.30} & \lr{97783.81} & \lr{2799259.98} & \lr{2}\\
				\hline
		\end{tabular}

	\label{tab:4gps_solve}
\end{table}
بر اساس جدول بالا، داده بدست آمده برای خطای ساعت غیرمنطقی است، پس، مساله جواب ندارد.


\subsection{بخش دوم}
در این بخش با انتخاب سه ماهواره و فرض $d = 0.05_{\sec}$ به حل سوال \lr{GPS} پرداخته شده است. در این بخش، همه‌ی حالت‌های ممکن (چهار حالت) بررسی شده است. محاسبات آن در فایل (\lr{Q3.m}) متلب آورده شده است. در هیچکدام از حالت‌ها این مساله جواب ندارد (برای بررسی فایل اشاره شده را اجرا کنید).
در ادامه، خطای موقعیت برابر است با عدم دقت ساعت در سرعت نور، فرض شده است.
\begin{equation}
	e = c \! \times \! t_e
\end{equation}
$$
e = c \! \times \! t_e \to 80 = 3 \! \times \! 10^8 t_e \to t_e = 26.7_{\mu \sec}
$$








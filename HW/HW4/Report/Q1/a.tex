\section{سوال اول}
در این سوال به بررسی مسیر بالستیک موشک و مسیر بهینه آن پرداخته شده است.
\subsection{بخش الف}
در این بخش به بررسی معادلات حرکت جسم نقطه در صفحه پرداخته شده است. معادلات حرکت جسم نقطه در صفحه به صورت زیر است:
\begin{equation}
    \label{eq:1}
    \ddot{\boldsymbol{\mathrm{r}}} = -\dfrac{GM}{r^3}\boldsymbol{\mathrm{r}} + \mathrm{Thrust} + \mathrm{Drag}
\end{equation}

در این معادله
$\mathrm{Thrust}$
نیروی پیشران موشک و
$\mathrm{Drag}$
نیروی مقاومت هوایی موشک است. در این مسیر فرض شده است که نیروی پیشران موشک به صورت زیر است:
\begin{equation}
    \label{eq:1-1}
    \mathrm{Thrust} = \dfrac{T}{v}\hat{\boldsymbol{\mathrm{v}}}
\end{equation}

در این معادله $T$ نیروی پیشران موشک و $m$ جرم موشک است. همچنین فرض شده است که نیروی مقاومت هوایی به صورت زیر است:
\begin{equation}
    \label{eq:1-2}
    \mathrm{Drag} = -\dfrac{1}{2}\rho C_D A v\hat{\boldsymbol{\mathrm{v}}}
\end{equation}

در این معادله $\rho$ چگالی هوا، $C_D$ ضریب مقاومت هوایی و $A$ مساحت مقطع عرضی موشک است. با جایگذاری معادلات \eqref{eq:1-1} و \eqref{eq:1-2} در معادله \eqref{eq:1} داریم:
\begin{equation}
    \label{eq:1-4}
    \ddot{\boldsymbol{\mathrm{r}}} = \left(-\dfrac{GM}{r^3}\boldsymbol{\mathrm{r}} + \dfrac{T}{m}\hat{\boldsymbol{\mathrm{v}}} - \dfrac{1}{2}\rho C_D A v\hat{\boldsymbol{\mathrm{v}}}/m\right)
\end{equation}

با توجه به اینکه در این مسیر فرض شده است که موشک در ارتفاع‌های بالا حرکت می‌کند، می‌توان فرض کرد که چگالی هوا تابعی از ارتفاع است. برای محاسبه چگالی هوا از رابطه زیر استفاده می‌شود:
\begin{itemize}
    \item $h \leq 0$ متر: $\rho = 1.225 \mathrm{kg/m^3}$
    \item $0 < h \leq 11000$ متر: $\rho = 1.225 \mathrm{kg/m^3} \left(1 - \dfrac{0.0065h}{288.15}\right)^{4.2561}$
    \item $11000 < h \leq 25000$ متر: $\rho = 0.36391 \mathrm{kg/m^3} \exp\left(\dfrac{-0.1577(h-11000)}{216.65}\right)$
    \item $25000 < h \leq 47000$ متر: $\rho = 0.08803 \mathrm{kg/m^3} \left(1 - \dfrac{0.0226(h-25000)}{216.65}\right)^{1.73}$
    \item $47000 < h \leq 53000$ متر: $\rho = 0.01322 \mathrm{kg/m^3} \exp\left(\dfrac{-0.1577(h-47000)}{216.65}\right)$
    \item $53000 < h \leq 79000$ متر: $\rho = 0.00143 \mathrm{kg/m^3} \left(1 - \dfrac{0.0065(h-53000)}{216.65}\right)^{4.2561}$
    \item $h > 79000$ متر: $\rho = 0$
\end{itemize}

پارامترهای معادله به صورت زیر تعریف می‌شوند:
\begin{itemize}
    \item $\boldsymbol{\mathrm{r}}$ : بردار موقعیت جسم نقطه
    \item $G$ : ثابت گرانشی
    \item $M$ : جرم جسم مرکزی
    \item $r$ : فاصله جسم نقطه از مرکز جسم مرکزی
\end{itemize}
بردار موقعیت جسم نقطه به صورت زیر تعریف می‌شود:
\begin{equation}
    \label{eq:2}
    \boldsymbol{\mathrm{r}} = x\hat{\boldsymbol{\mathrm{i}}} + y\hat{\boldsymbol{\mathrm{j}}}
\end{equation}
با جایگذاری معادله \eqref{eq:2} در معادله \eqref{eq:1} داریم:
\begin{equation}
    \label{eq:3}
    \ddot{x}\hat{\boldsymbol{\mathrm{i}}} + \ddot{y}\hat{\boldsymbol{\mathrm{j}}} = -\dfrac{GM}{(x^2 + y^2)^{3/2}}(x\hat{\boldsymbol{\mathrm{i}}} + y\hat{\boldsymbol{\mathrm{j}}}) + \dfrac{T}{m} (\dot x\hat{\boldsymbol{\mathrm{i}}} + \dot y\hat{\boldsymbol{\mathrm{j}}}) / v -\dfrac{1}{2}\rho C_D A v (\dot x\hat{\boldsymbol{\mathrm{i}}} + \dot y\hat{\boldsymbol{\mathrm{j}}})
\end{equation}
بر اساس روابط بالا ارتفاع به صورت زیر بدست می‌آید.
\begin{equation}
    \label{eq:4}
    h =\sqrt{x^2 + y^2} - a
\end{equation}
در این رابطه $a$ بیانگر شعاع زمین است. برای محاسبه سرعت 
تغیرات ارتفاع نیز به صورت زیر تعریف می‌شود.
\begin{equation}
    \label{eq:5}
    \dot{h} = \dfrac{x\dot{x} + y\dot{y}}{\sqrt{x^2 + y^2}}
\end{equation}
همچنین طول جغرافیایی برابر است با:
\begin{equation}
    \label{eq:6}
    \lambda = \tan^{-1}\left(\dfrac{y}{x}\right)
\end{equation}
و تغیرات طول جغرافیایی برابر است با:
\begin{equation}
    \label{eq:7}
    \dot{\lambda} = \dfrac{x\dot{y} - y\dot{x}}{x^2 + y^2}
\end{equation}
زاویه حمله به صورت زیر تعریف می‌شود:
\begin{equation}
    \label{eq:8}
    \gamma = \tan^{-1}\left(\dfrac{\dot{h}}{\dot{\lambda}}\right)
\end{equation}


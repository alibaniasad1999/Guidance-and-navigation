\section{سوال اول}
در این سوال به بررسی مسیر بالستیک موشک و مسیر بهینه آن پرداخته شده است.
\subsection{بخش الف}
در این بخش به بررسی معادلات حرکت جسم نقطه در صفحه پرداخته شده است. معادلات حرکت جسم نقطه در صفحه به صورت زیر است:
\begin{equation}
    \label{eq:1}
    \ddot{\boldsymbol{\mathrm{r}}} = -\dfrac{GM}{r^3}\boldsymbol{\mathrm{r}}
\end{equation}
پارامترهای معادله به صورت زیر تعریف می‌شوند:
\begin{itemize}
    \item $\boldsymbol{\mathrm{r}}$ : بردار موقعیت جسم نقطه
    \item $G$ : ثابت گرانشی
    \item $M$ : جرم جسم مرکزی
    \item $r$ : فاصله جسم نقطه از مرکز جسم مرکزی
\end{itemize}
بردار موقعیت جسم نقطه به صورت زیر تعریف می‌شود:
\begin{equation}
    \label{eq:2}
    \boldsymbol{\mathrm{r}} = x\hat{\boldsymbol{\mathrm{i}}} + y\hat{\boldsymbol{\mathrm{j}}}
\end{equation}
با جایگذاری معادله \eqref{eq:2} در معادله \eqref{eq:1} داریم:
\begin{equation}
    \label{eq:3}
    \ddot{x}\hat{\boldsymbol{\mathrm{i}}} + \ddot{y}\hat{\boldsymbol{\mathrm{j}}} = -\dfrac{GM}{(x^2 + y^2)^{3/2}}(x\hat{\boldsymbol{\mathrm{i}}} + y\hat{\boldsymbol{\mathrm{j}}})
\end{equation}
بر اساس روابط بالا ارتفاع به صورت زیر بدست می‌آید.
\begin{equation}
    \label{eq:4}
    h =\sqrt{x^2 + y^2} - a
\end{equation}
در این رابطه $a$ بیانگر شعاع زمین است. برای محاسبه سرعت 
تغیرات ارتفاع نیز به صورت زیر تعریف می‌شود.
\begin{equation}
    \label{eq:5}
    \dot{h} = \dfrac{x\dot{x} + y\dot{y}}{\sqrt{x^2 + y^2}}
\end{equation}
همچنین طول جغرافیایی برابر است با:
\begin{equation}
    \label{eq:6}
    \lambda = \tan^{-1}\left(\dfrac{y}{x}\right)
\end{equation}
و تغیرات طول جغرافیایی برابر است با:
\begin{equation}
    \label{eq:7}
    \dot{\lambda} = \dfrac{x\dot{y} - y\dot{x}}{x^2 + y^2}
\end{equation}
زاویه فراز با فرض قرار دادن محور  
$X$
بر روی مکان اولیه
به صورت زیر تعریف می‌شود:
\begin{equation}
    \label{eq:8}
    \theta = \arccos\left(\dfrac{\boldsymbol{\mathrm{r}}.\boldsymbol{r}_0}{r.r_0}\right) = \arccos\left(\dfrac{x_0x + y_0y}{\sqrt{(x^2 + y^2)(x_0^2 + y_0^2)}}\right)
\end{equation}
و تغیرات زاویه فراز برابر است با:
\begin{equation}
    \label{eq:9}
    \dot{\theta} = \dfrac{x_0\dot{x} + y_0\dot{y}}{\sqrt{(x^2 + y^2)(x_0^2 + y_0^2)}} - \dfrac{x_0x\dot{x} + y_0y\dot{y}}{(x^2 + y^2)^{3/2}\sqrt{x_0^2 + y_0^2}}
\end{equation}



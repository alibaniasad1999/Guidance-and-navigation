
% -------------------------------------------------------
%  Abstract
% -------------------------------------------------------


\pagestyle{empty}

\begin{وسط‌چین}
\مهم{چکیده}
\end{وسط‌چین}


\پرش‌بلند
\بدون‌تورفتگی 
در این مقاله، به یک مسئله اساسی در حرکت جمعی ربات‌های هوایی پرداخته شده است که چگونه مطمئن شویم که گروه‌های بزرگی از پهپادهای خودکار به صورت پیوسته در فضاهای محدود جابجا می‌شوند. مدل‌های گروه‌بندی موجود به ندرت بر روی سخت‌افزار واقعی آزمایش می‌شوند زیرا معمولاً برخی از جنبه‌های حیاتی سیستم‌های چندرباتی را نادیده می‌گیرند. حرکت و محدودیت‌های ارتباطی، تاخیرها، اختلالات یا وجود موانع باید به صورت صریح مدلسازی و درمان شوند زیرا تأثیر زیادی بر رفتار جمعی در همکاری عامل‌های واقعی دارند. دستکاری صحیح این مسائل منجر به پیچیدگی مدل اضافی و افزایش طبیعی تعداد پارامترهای تنظیم‌پذیر می‌شود، که نیازمند روش‌های بهینه‌سازی مناسب برای ارتباط محکم با توسعه مدل است. در این مقاله، مدلی از گروه‌بندی برای پهپادهای واقعی ارائه می‌دهیم که شامل یک چارچوب بهینه‌سازی تکاملی با پارامترها و توابع مناسب انتخاب شده است. به صورت عددی نشان داده شده است که رفتار گله ایجاد شده تحت شرایط واقعی برای اندازه‌های بزرگ گله و به ویژه با سرعت‌های بزرگ پایدار بود. نشان دادیم که الگوهای جمعی هماهنگ و واقع‌گرایانه حرکت در مقابل موانع ایجاد می‌شوند. علاوه بر این، مدل  در سخت‌افزار در حلقه اعتبارسنجی شده است.  آزمایش‌های میدانی با یک گروه خودسازماندهی از 30 پهپاد انجام شده است. این بزرگترین سیستم هوایی در فضای باز بدون کنترل مرکزی گزارش شده تا سال ۲۰۱۸ است که از جمعیت پرنده و جلوگیری اشتراکی بین آنها برخوردار است. نتایج، مناسب بودن رویکرد را تأیید کرده. کنترل موفقیت‌آمیز دسته‌هاي پهپادهاي چندمحوره به مدیریت وظایف بسیار کارآمدتر در زمینه‌های مختلفی که شامل پهپادها است، امکان خواهد داد.

 \مهم{کلیدواژه‌ها}: 
پهپاد، حرکت گله‌ای، هدایت، شبیه‌سازی، سخت‌افزار در حلقه، هدایت، حرکت جمعی، محیط محدود

\صفحه‌جدید

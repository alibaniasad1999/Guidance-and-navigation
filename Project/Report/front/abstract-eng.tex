
% -------------------------------------------------------
%  English Abstract
% -------------------------------------------------------


\pagestyle{empty}

\begin{latin}
	
	\begin{center}
		\textbf{Abstract}
	\end{center}
	\baselineskip=.8\baselineskip
	
	Multi-objective optimization is a challenging area in optimization due to the presence of multiple conflicting objectives that need to be optimized simultaneously. In the real world, many optimization problems come with various constraints, and traditional optimization algorithms may not be able to solve these problems effectively. In such scenarios, heuristic optimization algorithms play a crucial role in solving such problems.
	One such heuristic optimization algorithm is the REMARK algorithm, which is a random search method. The algorithm is known to be efficient in solving complex optimization problems with multiple objectives. The main advantage of the REMARK algorithm is that it can effectively handle the trade-off between multiple conflicting objectives. This is achieved by allowing population members to associate with each other, which leads to faster convergence and exploration of more susceptible locations in the search space.
	Moreover, the REMARK algorithm also ensures that each group of the population examines a part of the Pareto set, which allows for a high approximation of the set. This is particularly important in multi-objective optimization, as the Pareto set represents the set of non-dominated solutions that balance the conflicting objectives. The high approximation of the Pareto set achieved by the REMARK algorithm makes it a powerful tool for solving multi-objective optimization problems.
	In conclusion, multi-objective optimization presents significant challenges in the real world, but heuristic optimization algorithms like the REMARK algorithm provide a way to overcome these challenges. The REMARK algorithm is efficient, easy to implement, and provides a high approximation of the Pareto set, making it a useful tool in solving multi-objective optimization problems.
	
	\bigskip\noindent\textbf{Keywords}:Optimization algorithm, Multi-objective optimization, Population, Trading, Pareto set

	
\end{latin}

\newpage